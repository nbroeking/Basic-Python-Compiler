\documentclass{article}

\usepackage{graphicx}
\usepackage{minted}
\usepackage{color}
\usepackage{fancyhdr}
\usepackage[margin=1.0in]{geometry}
\usepackage{multicol}
\usepackage{titlesec}
\usepackage{filecontents}
\usepackage{url}

\definecolor{airforceblue}{rgb}{0.36, 0.54, 0.66}
\definecolor{secblue}{rgb}{0.18, 0.27, 0.33}

\newcommand{\backslashx}[0]{\textsc{Backslash x90 }}

\titleformat{\section}
  {\normalfont\scshape\color{secblue}}{\thesection}{1em}{}

\titleformat{\subsection}
  {\color{secblue}}{ \thesection}{1em}{}

\fancypagestyle{plain}{
    \fancyhf{}
    % \fancyhead[R]{\textcolor{blue}{\hrule}}
    \fancyhead[L]{\textcolor{airforceblue}{\textbf{\backslashx}}}
    \renewcommand{\headrulewidth}{2pt}% 2pt header rule
    \renewcommand{\headrule}{\hbox to\headwidth{%
        \color{airforceblue}\leaders\hrule height \headrulewidth\hfill}}
}

\usepackage[framemethod=TikZ]{mdframed}% Required for creating boxes
\mdfdefinestyle{left}{
    linecolor=gray, % Outer line color
%    frametitlerule=true, % Title rule - true or false
%    frametitlerulecolor=gray, % Title rule color
%    frametitlerulewidth=0.5pt % Title rule width
%    outerlinewidth=0.5pt, % Outer line width
%    roundcorner=0pt, % Amount of corner rounding
%    innertopmargin=0pt, % Top margin
%    innerbottommargin=0pt, % Bottom margin
%    innerrightmargin=0pt, % Right margin
%    innerleftmargin=10pt, % Left margin
%    leftmargin=12pt, % Left margin
%    backgroundcolor=white, % Box background color
%    frametitlebackgroundcolor=white, % Title background color
%    frametitlefont=\Large, % Title heading font specification
%    font=\small,
    leftline=false,
    topline=false,
    rightline=false,
    bottomline=false
}

\mdfsetup{nobreak=true}

\newcommand{\inputcode}[2]{
\begin{mdframed}[style=left,frametitle={\color{secblue}\emph{#1}}]
\begin{minipage}{\textwidth}
\inputminted{py}{#2}
\end{minipage}
\end{mdframed}
}


\pagestyle{plain}

\title{Implicit Parallelism at Runtime in Compiled Python Programs\\
    (Proposal)}
\author{Nicolas Broeking \and Joshua Rahm}

\begin{filecontents}{bib.bib}
@book{acm,
title = {Partitioned Global Address Space Languages.},
author = {Mattias De Wael, Stefan Marr, Bruno De Fraine, Tom Van Cutsem, Wolfgang De Meuter},
publisher = {ACCM Computing Surveys},
year = {2016},
}

@misc{delivery,
  author = {Christoph von Praun, Luis Ceze, Calin Cascaval},
  title = {{Implicit Parallelism with Ordered Transactions}},
  howpublished = "\url{https://homes.cs.washington.edu/~luisceze/publications/ipot_ppopp07.pdf}",
  note = "[Online; accessed 16-March-2015]"
}

@misc{dynamic,
  author = {Haiying Xu, Christopher J. F. Pickett, Clark Verbrugge},
  title = {{Dynmaic Purity Analysis for Java Programs}},
  howpublished = "\url{http://www.sable.mcgill.ca/~cpicke/papers/xu-07-dynamic.pdf}",
  note = "[Online; accessed 16-March-2015]"
}
\end{filecontents}


\begin{document}
\nocite{*}
\maketitle
\hrule

% \raggedcolumns
\begin{multicols}{2}
% \raggedcolumns

\begin{abstract}

As we make our way into the second half of the second decade of the third
millennium, Moore's law has finally begun to wane as processor clock speed have
started to stagnate. However, while processor clock speeds remain the same, the
number of cores in a processor has drastically increased. This is especially
true with the advent of general purpose GPUs (GPGPUs). How to use these cores
effectively is still a hard problem in software engineering especially because
parallel processing requires careful attention to protect against race
conditions and other caveats.  As a result, the approach many languages take is
to outlaw true parallel processing, but we believe that the answer is to bake
parallel processing into the emitted assembly code so that the programmer never
need to know of its existence, but may still enjoy the many benefits of a
multi-core machine.

\end{abstract}

\section{Group}
\backslashx is us; Nicolas Broeking and
Joshua Rahm. We are graduate students at the University of Colorado
Boulder with interests in Systems Development and Telecommunications.

\section{Problem} In software engineering, parallelism is still a very under
used featu
implement parallelism explicitly, such as C/C++ and Java, others, such as
Python and Ruby, must simply rely on the operating system's multiprocessing
facilities to manage multiple instances of the program and use the IPC
mechanisms of the operating system to share data.  Each of these approaches
have their benefits and their pitfalls.

In the latter case, it is well known that a context switch between two
different processes is \emph{much} slower than threads, not to mention
that passing data with messages via the operating system's IPC
interfaces in much slower than shared heap space. Yet still,
multiprocessing requires significant boiler plate programming to
implement. However, in spite of these shortcomings, it is easy to
implement this system once the boiler is in place, as synchronization is
not an issue.

In the former case, explicit threading, the problems are the inverse.
Manual threading is fast and efficient, but require due diligence to 
prevent race conditions and deadlocks. In addition, manual threading,
like its counterpart, still requires much boiler plate implementation
to use it to its full advantage, and yet the shared nature of threads
implies that if a single thread has a bug, then it may cause the entire
system to crash.

\section{Proposal}

We propose to go around the manual, explicit implementation and instead
let the compiler use a thread pool to parallelize problems implicitly.

The traditional computation model has been to have a single CPU execute
each instruction sequentially, but what if we are able to separate blocks
of execution in which they may be able to execute independently, and are
therefore able to be parallelized.

For example, take the simple Python code below.

\inputcode{Parallel Matrix}{example_1.py}

It is trivial to see that, in fact, $a$ and $b$ are actually independent of
each other, and as a result may be implemented independently.

Even though $m$ is assigned to, these can be semantically two different variables
and treated as such.

Another colloquial example is the following: 

\inputcode{Parallel List Comprehension}{example_2.py}

In this example if $f$ and $b$ are pure functions, meaning they do not have any
side effects, then we are able to parallelize this code for both aspects of this
list comprehension.

Similar purity analysis may be done on loops to determine if they are able to
be parallelized as well.

Finally, the most ambitious of the goals: being able to parallelize operations on
separate blocks of memory.


\inputcode{Partition Address Space}{example_3.py}
In this example, the parallelization properties are much more subtle; however, the
keen eye can see that the elements of $x$ and $y$ are changed independently of one
and other. This means we can rearrange this program to 

\inputcode{Partition Address Space 2}{example_4.py}

And we can then parallelize this as we did the first example.

\section{Approach}

Python is a dynamically typed programming language that naturally does in many
ways limit its ability to be effectively parallelized which will force us to
work with a subset of the Python Programming Language, $p_3$, that will allow us to
implement our improvements under the assumption that the user has not
overridden certain operators and types.

With these fairly limited changes to the Python language, we are still left with
the challenge of parallelizing a program which we have limited knowledge of at compile time.
Since this is the case, we will need to, at compile time, add many clues to indicate what
may be optimized and what may not be able to. This will add several phases to our compiler.

In order to accomplish our tasks we will extend the course compiler with two new phases, Purity Analysis and Interference Detection. 

\subsection*{$\vert$ Purity Analysis}

We may no longer assume that all functions will modify the global state of the program,
and discovering purity when it exists is going to be the main distinction between parallelizeable
code and sequential code. A handful of languages, most notably Haskell, have implemented
purity checking with their type checker. C++ has also to an extent with the $const$ keyword.

Detecting purity will be fairly straight forward. The function $f$ is considered pure if
it never invokes any impure functions. A non pure function is a function that modifies
the any variable that exists outside the scope of a function. So, for example, in this code
sample


\inputcode{Purity}{example_5.py}

The function $pure$ is in fact pure while the function $impure$ is impure. Even though,
both functions call $append$, a mutator function, $pure$ does it to a variable which
was created in its own scope.

The question now becomes, what if the $append$ function modifies some global state of the
program? It is already considered to be an impure function since it mutates $self$. This
is why, there needs to be some further analysis on member functions to determine if they are
globally pure (meaning they only mutate $self$) or if they are globally impure (they make
modifications outside of the scope).

This is going to require metadata at runtime to discover the levels at which aspects may
be parallelized. This data may be stored in a tag of the function object in the runtime to
indicate the level of purity of a function.

This means a function such as the following

\inputcode{}{example_6.py}

may be compiled to an intermediate language like

\inputcode{}{example_7.py}

\subsection*{$\vert$ Interference Detection}

To implement the second example where parallelism is achieved by looking at blocks of code
that do not interfere, we will solve a similar problem as register allocation. First
we will need to discover which parts of the code are separate from what other parts of the code.

Once again we will have a notion of purity. And we will know that all impure code will need
to be executed in their relative orders.

Because of the above, code section analysis must be done in three steps. First is separating pure
code from impure code. Second is detecting dependency relationships among the blocks of code.
Then, using this information, we can build an interference graph that show us what must be computed
sequentially, and what may be computed in parallel. Then, with our graph, we can trace the
execution of threads to optimally choose the number of threads to execute.

Separating pure and impure functions may be computed statement by statement and then each may be
globed together by their type. Once we have the impure and the pure code separated, we can detect
which blocks of code are dependent on others. All impure code block have an implicit temporal
dependency. Finally, code segments that do not interfere with each other may be parallelized.

\section{Literature Survey}

Our approach will add the ability, at runtime, detect what aspects of the code may be able to be parallelized In a dynamically typed language. The sources we have found have provided insights on how to implicitly parallelize a traditional, statically typed language, but we are attempting to extend their work to parallelize a dynamically typed language. The work done by HAL achieved parallelism by using
Partitioned Global Address Space Languages (PGAS). Another group, from IBM and The University of Illinois, achieved implicit parallelism
using ordered transactions (IPOT). Both of these frameworks provide many insights into the implementation of implicit parallelization; we
will use both of these insights in the design of our final compiler.

\section{Evaluation}

We will evaluate our success via a couple of metrics. First, we will consider our project successful
if we are able to successfully compile and be semantically correct while using parallel processing.

Better performance of the tests relative to the original compiler is not necessarily a requirement for
success, as we realize that of the tests are too small to gain much from multiple threads. However,
the project will be a great success if we are able to see significant improvement in the runtime
of our test suite.

\bibliographystyle{abbrv}
\bibliography{bib}

\vspace{\textheight}
\end{multicols}
\end{document}
