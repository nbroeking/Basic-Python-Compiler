\documentclass{acm_proc_article-sp}

\usepackage{lipsum}
\usepackage{minted}

\title{Implicit Parallelism in Compiled Python Programs}
\author{
    Nicolas Broeking \\
    \small Department of Computer Science \\
    \small University of Colorado Boulder \\
    \small \textbf{CSCI 5502} \\
    \small nicolas.broeking@colorado.edu \\
    \and
    Joshua Rahm \\
    \small Department of Computer Science \\
    \small University of Colorado Boulder \\
    \small \textbf{CSCI 5502} \\
    \small joshua.rahm@colorado.edu \\
}


\begin{document}

\maketitle

\begin{abstract}

In the second half of the current decade, Moore's law has
finally run out of steam as processor clock speeds have stagnated. However,
while processor clock speeds remain the same, the number of CPUs and processors
present on a device has drastically increased.  This is especially true with
the advent of general purpose GPUs (GPGPUs). How to effectively use these cores
is still a hard problem in software engineering especially because parallel
processing requires careful attention to protect against race conditions and
other caveats.  As a result, the approach many languages take is to outlaw true
parallelism, but we believe that the answer is to bake parallelism into the emitted
assembly code so that the programmer never need know of its existence, but may still
enjoy the many benefits of a multi-core machine.

\end{abstract}

\section*{Introduction}
When programming, we do not always use the machine in the most time efficient manner.
This is expressed mainly in programs that have parts which operate and execute independent
of each other but are forced by the language or the programmer to execute in sequence rather
than in parallel. We propose that instead of relying on the programmer to implement the threading,
we can allow the compiler to do some threading implicitly to avoid common errors programmers make
when trying to thread their applications.

We build this compiler such that it can detect if a function is pure; that is, a function
whose return value depends solely on the input supplied to it. Using this, we can parallelize
p3 in three stages.

First, we extend the runtime to have some more functions. These functions aid us in creating
the environment needed to do necessary threading at runtime. Second, we change the compiler to
emit the necessary calls to the extended runtime. And finally, we need to analyze the purity
of each function; is a function pure, impure or something in-between.

Using the steps from above, we successfully created an implicitly parallelizing compiler that
showed a significant increase in performance on many tests; all the while passing every
one of our 157 tests, including all the instructor's tests.

This project has left us with some features which we would like to see implemented, but do to
time constraints never got around to implementing, such as function scoring and implementing
a ``Green Thread'' model.

\section*{Problem}

There are many programs that can easily benefit from parallelism.
Consider the following program:

\inputminted{python}{pi.py}

In this program, we have several calls to \verb|calculatePi|, and this
function take 10 minutes to run, as a result this program takes over an hour
to run. This is most unfortunate, as this program is easily parallelized, just
the programmer decided not to implement any parallelism, and one cannot be blamed
for this decision. When implementing parallelism manually, not only is it anemic
in imperative languages, but is also full of potential pitfalls for the programmer
dealing with synchronization and race conditions.

If there is a way to make the machine detect when functions may be
parallelized and do the parallelism for the programmer, removing those potential
caveats and tedious laboring, there is the potential to create a
compiler that goes gangbusters.

For the Python programming language, implicit parallelism is not an easy
problem to solve. Some languages, like Haskell, already have implicit
parallelism, but this is under the structure of a very strong, static type
system that makes it much easier to detect when blocks of code and functions
may be able to run in parallel, all of which may be done at compile time. We
have the task of parallelizing python code which is much more dynamic and as
such, much of our logic for implementing the project must also be done
dynamically.


\section*{Implementation}

We extended the homework 6 compiler in 3 specific ways in order to achieve our
goals. First was extending the runtime to include functions for the compiled
programs to use. Second, was to emit the new assembly by extending the compiler
code itself, and finally, detecting pure functions.

\subsection*{1. Runtime}
We have extended the runtime to include two new public functions; \verb|dispatch| and
\verb|join|. The headers of which are shown below:

\begin{minted}{C}
u32_t dispatch(u32_t* out, pyobj func, int n, ...);
void join(u32_t);
\end{minted}

The function dispatch takes an address to write the return value to, a function
to call and a list of arguments passed as a \verb|va_list|. The return value of
dispatch is a thread id. This thread id is later used by the \verb|join| function
to know what thread to join on.

When dispatch is called, it does two things. First it checks to see if the
function passed is pure. This is done via metadata attached to the function at
compile time, a stage we will talk more about in the next section. If the
function is pure, \verb|dispatch| will then spawn a thread that runs a small
assembly routine \verb|__thread_start| that simply calls the function
\verb|func| with the arguments in the \verb|va_list| and stores the result in
the \verb|*out|. This thread id is then returned from the function to be joined
at a later time.

If the function is not pure, dispatch will just run the function sequentially
store the result in \verb|*out| and finally return 0, indicating the NULL
thread id.

Finally, \verb|join| is the analog to \verb|dispatch|. For every \verb|dispatch|,
there should be at least one matching \verb|join|. All join does is join on the thread
id if it is not 0 otherwise it does nothing and just returns.

\subsection*{2. Emit Runtime Calls}

Extending the compiler itself was the most difficult parts of this project.
First we realized that now, our return values had to be on the stack since
we needed to get the address to them. As a result we needed to add another rule
to our graph coloring called \verb|stack_only| meaning that the variable in
question had to be on the stack so we can take the address of it.

Once that step was completed, it was time for us to implement the famous \verb|leal| instruction
to get a pointer to the return value to pass as an argument to the \verb|dispatch|
function.

We ended up having to implement all the liveness and spill rules for this new instruction in order to remain semantically correct.

The liveness rules for \verb|leal| are

\[
L_{before}(leal(s,d)) = (L_{after} \cup \{s\}) \setminus \{d\}
\]

After adding support for the leal instruction, we are able to modify our callfunc
to emit assembly code that calls the runtime functions. For example, the program:

\begin{minted}{Python}
 def add(x, y):
	return x + y

x = add(1,2)
y = add(4, 6)
z = add(7,8)

print x
print z
print y
\end{minted}

will be preprocessed to semantically be equivalent to the following:

\begin{minted}{Python}
def add(x, y):
	return x + y
x, y, z
tid1 = dispatch(&x, add, 1, 2)
tid2 = dispatch(&y, add, 4, 6)
tid3 = dispatch(&z, add, 7, 8)

join(tid1)
print x
join(tid3)
print z
join(tid2)
print y
\end{minted}

In this new program, it is easy to see that instead of calling the functions directly,
we instead call the dispatch function with not only the original arguments, but also
a pointer to the return value (denoted by the C-style \verb|&|) and a closure to
call. Finally, before each value is used, the compiler is smart enough to inject a join
to ensure the thread calculating that value has stopped before the value is used. This
is a very valuable feature because it doesn't require that we stop the main thread until
we absolutly have to. 

\subsubsection*{Thread Liveness}

To implement the rule where each dispatch has at least one matching join, we
had to implement a sense of thread liveness. That is, in all branches of the
execution, we must make sure that the thread calculating a value is joined.

We do this in our flattening phase. As we iterate through each instruction, if
we see an assign from a \verb|CallFunc| then we add the left hand side of the
assignment to a list called \verb|joinable_vars|. At this point, if we see a
variable being used that is in the \verb|joinable_vars|, we inject a join
statement on that thread id before the usage of that variable and proceed to
remove that variable from the set of \verb|joinable_vars|. At this point the
Join node contains just the variable name, it is not until the next phase that
the variable names get mapped to thread ids.  If we come to an if statement, we
implement a conservative policy that propagates a copy of \verb|joinable_vars|
as we descend into the if statement. That way, the if statement does not affect
the \verb|joinable_vars| as we must assume that
no variables are joined.  We then emit code in the x86 that will join on the functions
outside of the If even if they have already been joined. In other words we didn't 
worry about joining functions multiple times. Doing it this way keeps our code semantically
correct and allows us to dramatically decrease the complexity of our compiler.

Similarly, loops must be assumed to not have been executed and as such,
any threads spawned outside the loop must also be joined outside the loop even
if they happened to be joined inside the loop.

For simplicity, we join all threads that haven't been joined at the end of each body of code.


For example, the following code snippet exemplifies this liveness

\begin{minted}{ Python }
z = input()
x = f(a)

y = 0
if z == 3:
    y = g(x + 2)

y = y + x + 1
\end{minted}

which is shown here

\begin{minted}{ Python }
z = input()
        {z}
x = f(a)
        {z, x}
y = 0
        <- Join(z)
        {x}
if z == 3:
            <- Join(x)
            {}
    y = g(x + 2)
            {y}
            <- Join(y)

        {x}
        <- Join(x)
        {}
y = y + x + 1
\end{minted}

\subsubsection*{Variable to Thread Mapping}

As we move to the register selection phase of our compiler, our internal Core
AST is filled with \verb|Join(x)| nodes where $x$ is the name of a variable.
What we need to know is change the variable names to actual thread ids. The way
we do this is, as we iterate through the Core AST, if we see an assign from a
function call, we add the left hand side of the assignment to a dictionary and
map it to a generated variable $t_x$ that represents the thread executing $x$.
Once we see a \verb|Join(x)|, we swap it with a $CallFunc(Join, t_x)$ node,
this then gets compiled into the correct instructions.


\subsection*{3. Purity Analysis}

Purity analysis is where a large portion of the work for this project went. We
detect the purity of a function during the uniquify and heapify phases of the
compiler. We walk through a function under the context that it is already pure
and look for evidence that the function is not pure. Rules for a pure function
are as follows:

\begin{enumerate} 

\item The pure function may not access its outer closure, as
this is subject to change and may produce impure results. (This includes
recursive functions) \item A pure function may not access the subscripts of any
arguments, or any local variable pointing to the arguments. As these are
subject to change with race conditions and lead to impure results.  \item A
pure function may only call a pure inner function.  \item A pure function may
not call print as this may lead to different print ordering, leading to impure
results.  \end{enumerate}

So some examples of impure functions would be 

\begin{minted}{Python}
def impure0(x):
    return x[0]

def impure1(x):
    def f():
        print "hi"
    f()
    return x

def impure2(x):
    return x + pure()
\end{minted}

An example of a pure function is:

\begin{minted}{Python}
def pure0(x):
    return x + 5

def pure1(x):
    lst = [1,2,3]
    return lst[0] + lst[1] + lst[2] +- x
\end{minted}

\subsubsection*{Conditional Purity}

During this project, we have coined a concept known as \emph{Conditional
Purity}. A conditionally pure function is a function that may be considered
pure depending on the inputs given to it as arguments. For example, consider
the following code:


\begin{minted}{Python}
def conditionally_pure(f, x, y):
    return f(x) + f(y)
\end{minted}

This function is considered impure by the rules stated above since it calls a function defined outside
its scope. This function; however, is pure if the function $f$ is pure. As such, this function is
conditionally pure on $f$. We can represent this conditionally pure attribute and at runtime detect
if the \emph{call} to the function is pure or impure.

\subsubsection*{Purity Marking}

In order to detect the purity of a function at runtime, we added an attribute to the function structure
that consists of a 32-bit bit mask called $purity$. This bit mask is equal to 0 for impure functions, 1 for unconditionally
pure functions, and, for conditionally pure function, $1 + \sum_{n=1}^{30}{e(n)2^{n+1}}$ where $n$ is the $n^{th}$
argument and $e(n)=1$ if the function is conditionally pure on the $n^{th}$ argument. $e(n)=0$ otherwise. In other
words, if the function is conditionally dependent on argument $n$, then the $n^{th} + 1$ bit is set to $1$.

The way the compiler stores this bit mask persistently is with a \verb|.long| tag that resides
in the \verb|.text| section before the definition of each function. For example, the functions
above produce the assembly:

\begin{minted}{gas}
.long 0
impure:
    pushl %ebp
    movl %ebp, %esp
    ...

.long 1
pure:
    pushl %ebp
    movl %ebp, %esp
    ...

.long 3 # binary 0000 ... 0011
conditionally_pure:
    pushl %ebp
    movl %ebp, %esp
    ...
\end{minted}

We modified create closure to take the value from the function pointer offset $-4$ to extract the tag
put by the compiler. This way, the dispatch function can retrieve all the information it needs to
determine if a function call can be parallelized at runtime.

\section*{Literature Survey}

The meat of our project was based off of a research project completed with IBM and the University of Illinois. This project
they were able to implement a Programming Model called Implicit Parallelism with Ordered Transactions or IPOT for short. 
IPOT is the idea that one can parallelize a sequential program using annotations. Each one of these annotations can be
threaded at the start of the annotation and then joined again when another transaction needs the data from the transaction. 
This gave us inspiration with how we wanted to structure our compiler.

Our execution model almost identically matches the IPOT execution model. It has a spawn / squash stage which is the equivalent to our dispatch and join runtime functions. Instead of a purity analysis phase though it uses a conflict detection phase. The main difference is that we perform analysis at a function by function level. However if you are able to more fully optimize the code by parallelizing chunks then you need to perform conflict detection. For the purposes of our project we decided to go with purity analysis instead.

While reviewing their paper, the use of annotations brought up an interesting question among our team. Is having the user annotate the blocks of code that can be parallelized truly implicit parallelism? We decided that no it isn't, and that we could extend their work by having the compiler do all the heavy lifting and create a truly implicit compiler.

\section*{Results}

\subsection*{Benchmarks}
We ran the new compiler with implicit parallelism baked into it on all the previous tests we made for
the compilers from the past and all passed. This shows that our compiler is still semantically correct.
We also benchmarked the tests against the compiler submitted for homework 6. Overall the compiler produced
code that ran $24.5\%$ slower than the homework 6 counterpart. This is mostly due to the extra overhead
of the dispatch function along with the overhead of spawning threads for simple functions, as our compiler
indiscriminately spawns threads for \emph{all} pure functions.

We did find very promising results for tailored examples run against the compiler. The following code was
run against both the homework 6 compiler and the new compiler:

\begin{minted}{Python}
def map_sum(f, x, y):                                                                                                                                                                                               
    return f(x) + f(y)                                                                                                                                                                                              
                                                                                                                                                                                                                    
def addOne(x):                                                                                                                                                                                                      
    y = 0                                                                                                                                                                                                           
    lst = [1,2,4]                                                                                                                                                                                                   
    while y != 10000000:                                                                                                                                                                                            
        y = y + lst[2] +- (lst[1] + lst[0])                                                                                                                                                                         
    y = y +- 9999999                                                                                                                                                                                                
    x = x + y                                                                                                                                                                                                       
    return x                                                                                                                                                                                                        
                                                                                                                                                                                                                    
x = map_sum(addOne, 4, 5)                                                                                                                                                                                           
y = map_sum(addOne, 5, 6)                                                                                                                                                                                           
z = map_sum(addOne, 7, 8)                                                                                                                                                                                           
w = map_sum(addOne, 9, 10)                                                                                                                                                                                          
v = map_sum(addOne, 11, 12)                                                                                                                                                                                         
                                                                                                                                                                                                                    
print x + y +- z + w +- v
\end{minted}

With this test case, we saw a 4x speedup with the new compiler and a 8x
speedup from interpreted Python on an AMD Phenom II processor. Notice that
this is with the \verb|map_sum| function being a conditionally pure function
and \verb|addOne| function being pure, so the compiler was able to detect purity
and conditional purity and the runtime was able to detect pure calls.

Even though this is a fairly tailored example, it does show promise into the potential
power of an implicitly parallelizing compilers and interpreters even if the language
is a dynamically typed language. 

\section*{Future Steps}
With the limited implementation time of this project, we were not able to implement everything
we wanted. Other possible extensions to the project are listed below.

\subsection*{Function Scoring}
The ability to score a function's complexity can eliminate the problem where
simple functions are unnecessarily spawned in a separate thread. The way function
scoring would have to be implemented is at runtime executing some form of static
analysis to try to estimate how many instructions will be executed on average by
this function. If the estimated number of instructions is more that the overhead
for starting a joining a thread, then it should be marked as can be parallelized.

\subsection*{Better Purity Rules}
Right now, our purity rules are Draconian. A future feature might be to liberalize
the purity rules by performing more static analysis to determine if, for example, 
calls to outer functions are provably pure or if free variables used are guaranteed
to be read only. This requires analysis not on the function itself, but all parent
function to make sure the variables read from have no chance of changing by the
functions that share the same closure.

\subsection*{Green Thread Model}

Languages like Haskell have a ``Green Thread'' model. What this means
is that instead of leveraging OS threads, these languages use user-space
``green'' threads. This makes spawning, joining and signaling green threads
much, much faster than the same operations for an OS thread. The disadvantage
is, of course, that green threads are not able to run on separate cores by
themselves. This is why languages like Haskell spawn one OS thread for each
CPU on the machine and then schedule the many user-space threads on these
separate OS threads. This allows for more efficient usage of processor resources
and reduces calls to the kernel to use the threads.

\subsection*{More Programmer Control}

Finally, sometimes the compiler should not be in charge of making decisions without
the programmers blessing. As such, it may be in the interests of the language designer
no allow additional annotations (like C's \verb|#pragma| or Java's \verb|@interface|) that
allow the programmer to specify what functions should be parallelize and what functions are
better left to run sequentially. After all, the programmer knows best how the code will
execute and whether it makes sense to have the compiler parallelize the code.


\section*{Conclusion}

We started out with the challenge to build a compiler for a dynamically typed
language and that we have done. We then challenged ourselves to make a compiler
that not only compiled Python, but also was able to implicitly parallelize it.
This is hard work seeing as even the Python interpreter does not even allow for
explicit parallelism. Nonetheless, we were able to put together a working prototype
that was able to parallelize simple programs.

We faced challenges with the dynamisms of Python, and realized there was no way
around doing much of the logic at runtime. Nonetheless, we were still able to
produce a compiler that generated code capable of spawning threads at appropriate time.

While our performance results at first may look discouraging as there was a general
slowdown. However, we did see a very significant boost in speed for some tailored case
which we believe implies that if the extensions mentioned above were to be implemented,
we will see a significant improvement in the performance of nearly all cases.

Overall, we consider the project to be a resounding success given the initial uncertainty
about whether or not it would even be possible to parallelize Python code, or even do
on the fly thread spawning and joining.


\end{document}
